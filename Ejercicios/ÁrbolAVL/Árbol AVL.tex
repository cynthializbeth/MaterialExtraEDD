\documentclass[12pt]{article} % Tamaño de letra
\usepackage[utf8]{inputenc} % Simbolo utf8
\usepackage[light]{antpolt} % Tipo de letra
\usepackage[T1]{fontenc}
\usepackage[spanish]{babel} % Español 
\usepackage[dvipsnames]{xcolor} % Paquete de colores
\usepackage{amsmath} % Paquete de simbolos matematicos
\usepackage{halloweenmath} % Paquete de Halloween
\usepackage{amssymb} % therefore 
\usepackage{amsfonts}  % Alinear
\usepackage{mathtools} 
\usepackage{hyperref} % Hiperreferencias
\usepackage{xurl} % Romper url largos en más líneas
\usepackage{graphicx}
\usepackage{caption} % Paquete para personalizar las captions
\usepackage{tikz} % Dibujos arreglo
\usepackage{graphics}
\usepackage{wrapfig}
\usepackage{subfig}
\usepackage{listings} % Código
\usepackage[shortlabels]{enumitem}
\usepackage{bbm}
\usepackage{multirow, array} % para las tablas
\usepackage{float}
\usepackage{color}
\usepackage{tabularray}
\usepackage{fancyhdr}
\usepackage{pifont}%Simbolos especiales.
\usepackage{fancybox}%Para crear cajas (Notas).
\usepackage[margin=1in]{geometry}
\usepackage[margin=1.1in]{caption}
\pagestyle{fancy}
\fancypagestyle{plain}{
\fancyhf{}\fancyhead[L]{\includegraphics[height=0.5in]{unam.png}}
\fancyhead[R]{\includegraphics[height=0.5in]{ciencias.png}}
}

%----------[ Declaración Respuesta ]-----------
\usepackage{framed,xcolor} % Color de repuesta

\newenvironment{respuesta}{
    \definecolor{shadecolor}{RGB}{236,236,228}
    \begin{snugshade*}
    \vspace*{2mm}}
    {\vspace*{2mm}
    \end{snugshade*}}

%------------------------------
\definecolor{fondo}{RGB}{234, 236, 238}% Definición del color fondo

\lstset{
    language=[Sharp]C,
    basicstyle=\ttfamily\footnotesize, % Cambiar el tamaño de la fuente
    keywordstyle=\color{blue},
    commentstyle=\color{OrangeRed},
    stringstyle=\color{ForestGreen},
    numbers=left,
    numberstyle=\tiny,
    stepnumber=1,
    numbersep=5pt,
    backgroundcolor=\color{fondo},
    frame=single,
    rulecolor=\color{black},
    breaklines=true,
    columns=flexible,
    inputencoding=utf8,
    %morecomment=[s][\color{purple}]{/**}{*/},
    %morecomment=[l][\color{orange}]{//},
    literate={\#}{{\#}}1,
}

\begin{document}
\title{Ejercicios 5: Árbol AVL}
\author{Ayudante: Cynthia Lizbeth Sánchez Urbano}
\date{Ejercicios para repaso de Árboles AVL.}
\maketitle
\begin{enumerate}
\item Dibuja un nodo AVL con sus atributos.
\item ¿Qué distingue a un árbol AVL de un árbol binario ordenado?
\item ¿Qué es la altura de un nodo y cómo se calcula?
\item ¿Cómo se obtiene el balance de un nodo?
\item Explica cómo se realiza un giro sobre un nodo a la izquierda.
\item En que casos debemos girar dos veces.
\item Crea el árbol AVL que contenga los elementos 7, 9, 6, 3, 5, 46, 20, 5. Coloca las alturas y el balance de cada nodo.
\item Al árbol de la pregunta 7 eliminale el 20.
\item Al árbol resultante eliminale el 5.
\item Crea un diagrama en donde muestres la relación entre los siguientes tipos de árboles:
\begin{itemize}
\item Árbol Rojinegro
\item Árbol Completo
\item Árbol AVL
\item Árbol con todos los niveles llenos (Lleno)
\item Árbol Binario Ordenado
\item Árbol Binario
\end{itemize}
\end{enumerate}
\end{document}